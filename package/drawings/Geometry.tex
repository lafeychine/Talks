\usetikzlibrary{3d, angles, shapes, shapes.misc}

\tikzset{
    Grid/.style={ultra thin, color=lightgray, opacity=0.8},
    Vector/.style={midway, sloped, above},
    Angle/.style={draw, angle radius=.5cm}
}

\tikzset{
    cross/.style={cross out, thick, draw=black, fill=none, minimum size=2*(#1-\pgflinewidth), inner sep=0pt, outer sep=0pt},
    cross/.default={3pt}
}

\NewDocumentCommand \drawGrid {s D<>{1-} O{Grid} m m m m} {
    \IfBooleanT #1 {
        \draw<#2> [#3] (#4 + .1, #5 + .1) grid (#6 - .1, #7 - .1);
    }

    \draw<#2>[thick, ->] (#4, 0)  -- (#6, 0)  node [above] {$ x $};
    \draw<#2>[thick, ->] (0,  #5) -- (0,  #7) node [left] {$ y $};
}


\NewDocumentCommand \drawGridSpace {s D<>{1-} m m m} {
    \IfBooleanT #1 {
        \begin{scope}[canvas is xy plane at z=0]
            \draw<#2> [Grid] (0, 0) grid (#3 - .1, #4 - .1);
            \draw<#2> (.3, 0, 0) |- (0, .3, 0);
        \end{scope}

        \begin{scope}[canvas is yz plane at x=0]
            \draw<#2> [Grid] (0, 0) grid (#4 - .1, #5 - .1);
            \draw<#2> (.3, 0, 0) |- (0, .3, 0);
        \end{scope}

        \begin{scope}[canvas is xz plane at y=0]
            \draw<#2> [Grid] (0, 0) grid (#3 - .1, #5 - .1);
            \draw<#2> (.3, 0, 0) |- (0, .3, 0);
        \end{scope}
    }

    \draw<#2>[thick, ->] (0, 0, 0) -- (#3, 0, 0) node [right] {$ x $};
    \draw<#2>[thick, ->] (0, 0, 0) -- (0, #4, 0) node [above] {$ y $};
    \draw<#2>[thick, ->] (0, 0, 0) -- (0, 0, #5) node [below left] {$ z $};
}


\NewDocumentCommand \drawPoint {s D<>{1-} O{above right} r() m m D(){#4}} {
      \coordinate (#4) at (#5, #6);

      \IfBooleanT #1 {
          \draw<#2>[dashed] (#5, 0) -- (#5, #6);
          \draw<#2>[dashed] (0, #6) -- (#5, #6);
      }

      \draw<#2> (#4) node[cross] {} node[#3] {$ #7 $};
}


\NewDocumentCommand \drawPointSpace {s D<>{1-} O{above right} r() m m m D(){#4}} {
      \coordinate (#4) at (#5, #6, #7);

      \IfBooleanT #1 {
          \draw<#2>[dashed] (#5, 0, 0)  -- (#5, 0,  #7);
          \draw<#2>[dashed] (0,  0, #7) -- (#5, 0,  #7);
          \draw<#2>[dashed] (#5, 0, #7) -- (#5, #6, #7);
      }

      \draw<#2> (#4) node[cross] {} node[#3] {$ #8 $};
}
